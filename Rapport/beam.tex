\documentclass{beamer}
\mode<presentation>

\usetheme{Pittsburgh}
\usecolortheme{seahorse}
\usecolortheme{rose}

\usepackage[francais]{babel}
\usepackage[utf8]{inputenc}
\usepackage[T1]{fontenc}

\usepackage{semantic}

\usepackage{listings, multicol}
\lstset{language=[Objective]Caml, captionpos=b, frame=single}

\newcommand\heading[1]{%
  \par\bigskip
  {\Large\bfseries#1}\par\smallskip}

\newcommand{\sslash}{\mathbin{/\mkern-6mu/}}

%% Hack for minus signs in listings....
\mathchardef\hyphenmathcode=\mathcode`\-
\let\origlstlisting=\lstlisting
\let\endoriglstlisting=\endlstlisting
\renewenvironment{lstlisting}
{\mathcode`\-=\hyphenmathcode
    \everymath{}\mathsurround=0pt\origlstlisting}
{\endoriglstlisting}

%% Déclaration des fonctions mathématiques
\DeclareMathOperator{\iT}{T}
\DeclareMathOperator{\iH}{H}
\DeclareMathOperator{\iTI}{TI}
\DeclareMathOperator{\iDom}{dom}
\DeclareMathOperator{\iLambda}{\lambda}
\DeclareMathOperator{\ienvSPlit}{envSplitsN}
\DeclareMathOperator{\isplit}{splitsN}

\title{Implémentation  d'un algorithme d'\emph{inhabitation} en OCaml}
\subtitle{Compte rendu de TRE}
\author{Ulysse Gérard}
\institute{Université Denis Diderot, Paris 7\\TRE encadré par Delia Kesner (PPS)}
\date{9 septembre 2014}

\begin{document}

    
    \begin{frame}
        \titlepage
    \end{frame}
    
    \begin{frame}{Sommaire}
        \tableofcontents
    \end{frame}
    
    \section{Introduction}
    \begin{frame}{Introduction}
        \heading{Le $\lambda$-calcul}
        Un système formel, Turing-complet, programmation fonctionnelle
        \begin{block}{Le problème d'\emph{inhabitation}}
        Il s'agit de répondre a la question de l'existence, pour un type $\tau$ et un environnent de typage $\Gamma$ donnés, d'un terme $t$ tel que $\Gamma\vdash t:\tau$
        \end{block}
        \heading{Types simples}
        \begin{itemize}
            \item Le problème de l'\emph{inhabitation} est décidable \cite{Ben}
            \item $t$ typable $\Rightarrow$ $t$ fortement normalisable (ie: toutes les réductions à partir de $t$ sont finies ($(\lambda x.xx)(\lambda x.xx))$)
            \item $t$ typable $\nLeftarrow$ $t$ fortement normalisable ($\lambda x.xx$)
        \end{itemize}
    \end{frame}
        
    \begin{frame}{Introduction}
    \heading{Types intersection}
    \begin{itemize}
    \item Assigner des types multiples à un même terme
    \item Caractérisation possible de la forte normalisation avec notamment le système $\mathcal{D}\Omega$ \cite{coppo}
        \begin{itemize}
        \item Utilise types intersection \emph{associatifs} ($(A\wedge B)\wedge C = A\wedge (B\wedge C)$), \emph{commutatifs} ($A\wedge B=B\wedge A$) et \emph{idempotents} ($A\wedge A = A$)
        \item Prouvé indécidable (Urzyczyn \cite{Urz})
        \end{itemize}
    
    
    \item Vaincre l'indécidabilité: types intersection \emph{non-idempotents} ($A\wedge A \neq A$)
    \end{itemize}
    \end{frame}
    
    \begin{frame}
    \heading{Types intersection non-idempotents}
    \begin{itemize}
        \item Travaux récents de chercheurs du PPS et de l'université de Turin.
        \item Un autre système de types, le système $\mathcal{M}$
        \item Ils ont construit un algorithme d'\emph{inhabitation} complet, correct et qui termine, prouvant ainsi la décidabilité du problème.
        \item Objectif de ce TRE: implémenter cet algorithme en OCaml
    \end{itemize}
    \end{frame}
    
    
    \section{Présentation du formalisme et de l'algorithme}
    \begin{frame}{Formalisme}
        \begin{block}{Le $\lambda$-calcul}
        \vspace{-.3cm}
           \begin{equation*}
           t, u, v ::= x\ |\ \lambda x.t\ |\ tu 
           \end{equation*}	
        \end{block}
        
        \begin{definition}[$\alpha$-conversion]
            Tous les termes du $\lambda$-calcul sont définis \emph{modulo $\alpha$-conversion}, c'est à dire que l'on peut toujours remplacer une variable liée par une autre non-encore utilisée.
            
            \begin{equation*}
            \text{Ainsi: } \lambda x.t =_\alpha \lambda y.t[x\sslash y] \text{ pour $y$ fraîche}
            \end{equation*}
        \end{definition}
        \begin{definition}[$\beta$-réduction]
            \vspace{-.3cm}
            \begin{equation*}
              (\lambda x.t) u ->_\beta t\{x/u\} \qquad\text{ (Ex: $(\lambda x.tx)u \rightarrow_\beta tu$)}
            \end{equation*}
        \end{definition}
    \end{frame}

    \begin{frame}
    \begin{block}{System $\mathcal{M}$}
        \begin{gather*}
                \inference[(Var)]{}{x : [\rho] \vdash x : \rho}
                \qquad
                \inference[($\rightarrow$ I)]{\Gamma, x : A \vdash t : \rho}{\Gamma \backslash x \vdash \lambda x.t : A \rightarrow \rho}
                \\ \\
                \inference[($\rightarrow$ E)]{\Gamma_0 \vdash t : [\sigma_1, \ldots \sigma_i] \rightarrow \rho 
                    & \Gamma_1 \vdash u : \sigma_1, \ldots \Gamma_i \vdash u : \sigma_i}
                {\Gamma_0 + \Gamma_1 + \cdots + \Gamma_i \vdash tu : \rho}
                \\ \\
                \text{Pour les types } \sigma, \tau, \rho ::= \alpha \ | \ A \rightarrow \tau \qquad A ::= [\sigma_i]_{i \in I}
                \\ \text{Et où pout tout } x \in \text{domain}(\Gamma_0)\cup\text{domain}(\Gamma_1) ,
                 \\  (\Gamma_0 + \Gamma_1)(x) = \Gamma_0(x) \uplus\Gamma_1(x) 
        \end{gather*}
    \end{block}
    \end{frame}
    
    \begin{frame}{Système $\mathcal{M}$}
    \begin{itemize}
    \item Ce système est dit "relevant": la règle \texttt{(var)} ne s'applique qu'avec un environnement minimal
    \item Peux typer des termes non fortement normalisables:
        \begin{displaymath}
            \inference[($\rightarrow$ E)]{x:[[] \rightarrow \alpha] \vdash x : [] \rightarrow \alpha}{x : [[] \rightarrow \alpha] \vdash x((\lambda y.yy)(\lambda y.yy)) : \alpha}
        \end{displaymath}
    \item Pour représenter ces sous-termes non-typables on utilise des formes normales approximées (anf):
    \begin{align*}
        a,b,c ::= \Omega\ |\ \mathcal{N} 
        && \mathcal{N} ::= \lambda x.\mathcal{N}\ |\ \mathcal{L}
        && \mathcal{L} ::= x\ |\ \mathcal{L}a
    \end{align*}
            Ordre sur les anf: $\lambda x.x\Omega\Omega \leq \lambda x.x\Omega y \leq \lambda x.xzy$
    \end{itemize}
    \end{frame}
    
%    \begin{frame}{L'algorithme}
%    
%    \begin{itemize}
%    \item $\iT(\Gamma, \sigma)$ contient toutes les anf $a$ telles qu'il existe une dérivation $\Pi \triangleright \Gamma \vdash a : \sigma$ pour laquelle $a$ est minimale. 
%    \item $\iH^\Delta_a(\Gamma, \sigma) \triangleright \tau$ contient toutes les anf $b$ telles que $a$ soit un sous-terme de $b$, et tel que  $a \in \iT(\Delta, \sigma) \Rightarrow b \in \iT(\Gamma + \Delta, \tau)$.
%    \item $\iTI(\Gamma, [\sigma_i]_{i \in I})$ contient toutes les anf $a = \bigvee_{i\in I}a_i$ telles que  $ \Gamma = +_{i \in I} \Gamma_i $, $ a_i \in \iT(\Gamma_i, \sigma_i)\text{ pour tout }i \in I $ et $\uparrow_{i \in I}a_i$
%    \end{itemize}
%    \end{frame}
    
    \begin{frame}{L'algorithme}
        \begin{gather*}
        \inference[(ABS)]{a \in \iT(\Gamma + (x : A), \tau) & x \notin \iDom(\Gamma)}{\lambda x.a \in \iT(\Gamma, A \rightarrow \tau)}
        \\ \\
        \inference[(HEAD)]{a \in \iH^{x:[\sigma]}_x(\Gamma, \sigma) \triangleright \tau}{a \in\iT(\Gamma + (x:[\sigma]), \tau) }
        \\ \\
        \inference[(PREFIX)]{\Gamma = \Gamma_0 + \Gamma_1 & b \in \iTI(\Gamma_0, A) & a \in \iH^{\Delta + \Gamma_0}_{cb}(\Gamma_1, \sigma) \triangleright \tau}{a \in \iH^{\Delta}_c(\Gamma, A \rightarrow \sigma) \triangleright \tau}
        \\ \\
        \inference[(FINAL)]{\sigma = \tau}{a \in \iH^{\Delta}_a(\emptyset, \sigma) \triangleright \tau}
        \qquad
        \inference[(UNION)]{(a_i \in \iT(\Gamma_i, \sigma_i))_{i \in I} & \uparrow_{i \in I}a_i}{\displaystyle \bigvee_{i \in I} a_i \in \iTI(+_{i \in I}\Gamma_i, [\sigma_i]_{i \in I}) }
        \end{gather*}
    \end{frame}
    
    \begin{frame}
        \begin{displaymath}
        \makebox[\textwidth][c]{
        \inference[U]
        {\inference[Head]
            {\inference[Final]
                {\alpha = \alpha}
                {y \in \iH^{y:[\alpha]}_y(\emptyset, \alpha) \triangleright \alpha}
            }
            {y \in \iT(\{y:[\alpha]\}, \alpha)}
        }
        {y \in \iTI(\{y:[\alpha]\}, [\alpha])}
        
        \inference[F]
        {\alpha = \alpha}
        {xy \in \iH^{\{x:[[\alpha]->\alpha], y : [\alpha]\}}_{xc}(\emptyset, \alpha) \triangleright \alpha}
    }
        \end{displaymath}
            \begin{displaymath}
            \makebox[\textwidth][c]{
            
        
            \inference[Abs]
            {\inference[Abs]
                {\inference[Head]
                    {\inference[Prefix]
                        {
                        }
                        {xy \in \iH^{x : [[\alpha] -> \alpha]}_x(\{y:[\alpha]\}, [\alpha] -> \alpha) \triangleright \alpha}
                    }
                    {xy \in \iT(\{x : [[\alpha] -> \alpha], y : [\alpha]\}, \alpha)} \\\text{and } y \notin \iDom(\{x : [[\alpha] -> \alpha]\})}
                {\lambda y.xy \in T(\{x : [[\alpha] -> \alpha]\}, [\alpha] -> \alpha)} \\\text{and } x \notin \iDom(\emptyset)}
            {\lambda x.\lambda y.xy \in \iT(\emptyset, [[\alpha] -> \alpha] -> [\alpha] -> \alpha)}
        
        }
        \end{displaymath}
    \end{frame}
    
    
    \section{Implémentation en OCaml}
    \begin{frame}[containsverbatim]{Implémentation}
        \begin{itemize}
            \item Un algorithme fortement non-déterministe
            \item Très adapté à la programmation récursive
        \end{itemize}
        
        \begin{multicols}{2}
        \begin{lstlisting}
type 'a multiset =
 Empty
|Cons of 'a*'a multiset
        \end{lstlisting}
\begin{lstlisting} %[caption={Intersection types and environments}]

type multisetType = 
 sType multiset 
and sType = 
 Var of char
|Fleche of multisetType
            * sType
\end{lstlisting}

        \vfill
        \columnbreak
        \begin{lstlisting}
type anf = 
    Omega
  | N of n
and n = 
    Lambda of char * n
  | L of l
and l =
    Var of int
  | App of l * anf
            \end{lstlisting}
            
        \end{multicols}
\begin{lstlisting}
type environment = (char * multisetType) list
\end{lstlisting}
    \end{frame}
    
    \begin{frame}[containsverbatim]
    \heading{La fonction principale, $\iT$}
    \begin{lstlisting}
let rec t env type0 fresh =
  match env, type0 with
      ([], Fleche(type1, type2)) -> ABS
    | (_, Fleche(type1, type2)) -> ABS@HEAD
    | (_::_, _) -> HEAD
    | (_,_) -> []
    \end{lstlisting}
    
    \begin{multicols}{2}
    \heading{Abs}
    \columnbreak
    \heading{Head}
    \end{multicols}
    \begin{multicols}{2}
    \small
    \begin{displaymath}
        \inference[]{a \in \iT(\Gamma + (x : A), \tau) & x \notin \iDom(\Gamma)}{\iLambda x.a \in \iT(\Gamma, A \rightarrow \tau)}
    \end{displaymath}
    \vfill
    \columnbreak
    \begin{displaymath}
        \inference[]{a \in \iH^{x:[\sigma]}_x(\Gamma, \sigma) \triangleright \tau}{a \in\iT(\Gamma + (x:[\sigma]), \tau) }
    \end{displaymath}
        \end{multicols}
        \normalsize
    \end{frame}
    
    
    \begin{frame}[containsverbatim]
        \heading{La fonction $\iH$}
        \begin{lstlisting}
let rec h env1 env2 lf type1 type2 fresh =
  match env2, type1 with
    ([], Fleche(typeA, typeS)) -> FINAL@PREFIX
    | ([], type1) -> FINAL
    | (_, Fleche(typeA, typeS)) -> PREFIX
    | (_, _) -> []
        \end{lstlisting}
        \heading{Prefix}
        \begin{displaymath}
        \inference[]{\Gamma = \Gamma_0 + \Gamma_1 & b \in \iTI(\Gamma_0, A) & a \in \iH^{\Delta + \Gamma_0}_{cb}(\Gamma_1, \sigma) \triangleright \tau}{a \in \iH^{\Delta}_c(\Gamma, A \rightarrow \sigma) \triangleright \tau}
        \end{displaymath}
        
        \heading{Final}
        \begin{displaymath}
        \inference[]{\sigma = \tau}{a \in \iH^{\Delta}_a(\emptyset, \sigma) \triangleright \tau}
        \end{displaymath}
    \end{frame}
    
    \begin{frame}
            \heading{Partitions d'un environnement}
            Exemple de découpe en deux de  $\Gamma = \{(x : [\alpha \rightarrow \alpha, \alpha]), (y : [\alpha])\}$:
            \begin{align*}
               \Gamma_0 &= \Gamma &&and&& \Gamma_1 = \emptyset
               \\\Gamma_0 &= \{(x:[\alpha]), (y:[\alpha])\} &&and&& \Gamma_1 = \{(x:[\alpha -> \alpha])\}
               \\\Gamma_0 &= \{(y:[\alpha])\} &&and&& \Gamma_1 = \{(x : [\alpha \rightarrow \alpha, \alpha])\}
               \\\Gamma_0 &= \{(x : [\alpha \rightarrow \alpha, \alpha])\} &&and&& \Gamma_1 = \{(y:[\alpha])\}
               \\&&&\vdots&&
               \\\Gamma_0 &= \emptyset &&and&& \Gamma_1 = \Gamma
            \end{align*}
            
            Illustration d'un découpage en 3 de $\{a,b,c\}$:
            \begin{align*}
                &\ [\ []\ ,\ []\ ,\ []\ ]\ &
                \\&\ [\ [\ [c]\ ,\ []\ ,\ []\ ]\ ,\ [\ []\ ,\ [c]\ ,\ []\ ]\ ,\ [\ []\ ,\ []\ ,\ [c]\ ]\ ]\ &
                \\&\ [\ [\ [cb]\ ,\ []\ ,\ []\ ]\ ,\ [\ [c]\ ,\ [b]\ ,\ []\ ]\ ,\ [\ [c]\ ,\ []\ ,\ [b]\ ]\ ,
                \ [\ [b]\ ,\ [c]\ ,\ []\ ]\ ,&
                \\&\ \ \ [\ []\ ,\ [cb]\ ,\ []\ ]\ ,\ [\ []\ ,\ [c]\ ,\ [b]\ ]\ ,
                \ [\ [b]\ ,\ []\ ,\ [c]\ ]\ ,\ [\ []\ ,\ [b]\ ,\ [c]\ ]\ ,\ [\ []\ ,\ []\ ,\ [cb]\ ]\  ]&
                \\&\ldots&
            \end{align*}
    \end{frame}
    
    \begin{frame}[containsverbatim]
    \heading{La fonction $\iTI$}
    \begin{lstlisting}
 val union: env -> multisetType -> int -> anf
    \end{lstlisting}
    
    \begin{displaymath}
        \inference[]{(a_i \in \iT(\Gamma_i, \sigma_i))_{i \in I} & \uparrow_{i \in I}a_i}{ \bigvee_{i \in I} a_i \in \iTI(+_{i \in I}\Gamma_i, [\sigma_i]_{i \in I}) }
    \end{displaymath}
    
    \heading{Calcul de la borne supérieure}
    \begin{gather*}
        \text{For all terms }t,\ \Omega \vee t = t \vee \Omega = t 
        \\
        \text{For all terms }t_{1..4},\ t_1t_2\vee t_3t_4 = (t_1\vee t_3)(t_2\vee t_4) \\
        \text{For all variable $x$ and terms $t$ and $u$, } \lambda x.t \vee \lambda x.u = \lambda x.(tu)
    \end{gather*}
    
    \end{frame}
    
    \begin{frame}[Exemples d'exécution]
   
   \begin{enumerate}
%       \item $\Gamma = \emptyset$ and $\sigma = \alpha$: No results
       \item $\Gamma = \{(x:[\beta])\}$ and $\sigma = \alpha$: No results
       \item $\Gamma = \{(x:[\alpha])\}$ and $\sigma = \alpha$:
           \qquad\texttt{Head Final} $->x$ 
    \pause
       \item $\Gamma = \emptyset$ and $\sigma = [\alpha] -> \alpha$:
           \qquad \texttt{Abs (Head Final)} $->\lambda x.x$ 
       \pause
              \item $\Gamma = \emptyset$ and $\sigma = [\alpha,\ [\alpha] -> \alpha] -> \alpha$: \\
                   \texttt{Abs (Head (Prefix (Union (Head Final), Final)))}  $->\lambda x.xx$
        \pause
       \item $\Gamma = \emptyset$ and $\sigma = [[] -> \alpha] -> \alpha$:\\
           \qquad \texttt{Abs (Head (Prefix (Union End, Final)))}  $->\lambda x.x\Omega$
       
       \pause
%       \item $\Gamma = \{(x:[[\alpha] -> \alpha])\}$ and $\sigma = \alpha$: No results 
       
%       \item $\Gamma = \{(x:[\alpha,\ [\alpha] -> \alpha])\}$ and $\sigma = \alpha$: \\
%           \qquad \texttt{Head (Prefix (Union (Head Final), Final))}  $->xx$ 
       
       \item $\Gamma = \{(x:[[\alpha] -> \alpha])\}$ and $\sigma = [\alpha] -> \alpha$: \\
           \qquad \texttt{Head Final} $->x$\\
            \texttt{Abs (Head (Prefix (Union (Head Final), Final)))} $->\lambda y.xy$
       \pause
%%       \item $\Gamma = \{(x:[[\alpha] -> \alpha]),\ (x':[[\alpha] -> \alpha]),\ y:[\alpha]\}$ and $\sigma = [\alpha]$:\\
%           \qquad \texttt{Head (Prefix (Union (Head (Prefix (Union (Head Final), Final))), Final))} $-> x(x'y)$\\
%           \qquad \texttt{Head (Prefix (Union (Head (Prefix (Union (Head Final), Final))), Final))} $-> x'(xy)$\\
           
       \item $\Gamma = \emptyset$ and $\sigma = [[\alpha] -> \alpha] -> [\alpha] -> \alpha$ : \\
           \qquad \texttt{Abs (Head Final)} $->\lambda x.x$\\
           \qquad \texttt{Abs (Abs (Head (Prefix (Union (Head Final), Final))))} $->\lambda xy.xy$    
   \end{enumerate}
    \end{frame}
    
    \section{Conclusion}
    \begin{frame}{Conclusion}
    \begin{itemize}
        \item Des types intersection non-idempotents pour un algorithme d'\emph{inhabitation} décidable
        \item Une implémentation en OCaml tirant parti de la récursivité
        \item Mais une complexité possiblement très grande, notamment en raison de l'utilisation massive de méthodes non récursives terminales, comme \texttt{List.map}
        \item Aller plus loin: \emph{inhabitation} pour le $\lambda$-calcul avec patterns
    \end{itemize}
    \end{frame}
    
    \begin{frame}{Biliographie}
    \bibliographystyle{plain}
    \bibliography{main.bib}
    \end{frame}
    
\end{document}
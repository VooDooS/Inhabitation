\documentclass{article}

\usepackage[french, english]{babel}
\usepackage[utf8]{inputenc}
\usepackage[T1]{fontenc}

\usepackage{mdframed}
\usepackage{xcolor}

%% Better maths
\usepackage{amsmath, amssymb, amsthm, xfrac, semantic}

%% Code listings
\usepackage{listings}
\lstset{language=[Objective]Caml, frame=none, captionpos=b, frame=single}

\usepackage{multicol}

%% Déclaration des fonctions mathématiques
\DeclareMathOperator{\iT}{T}
\DeclareMathOperator{\iH}{H}
\DeclareMathOperator{\iTI}{TI}
\DeclareMathOperator{\iDom}{dom}
\DeclareMathOperator{\iLambda}{\lambda}
\DeclareMathOperator{\ienvSPlit}{envSplitsN}
\DeclareMathOperator{\isplit}{splitsN}

%% Environnements maths (amsthm)
\newtheorem{Def}{Definition}

\def\lcalc{$\lambda$-calculus}
\def\lci{$\lambda$-calculi}
\def\lterm{$\lambda$-term}
\def\lterms{$\lambda$-terms}
\def\sysm{system $\mathcal{M}$}


\title{\textbf{Inhabitation Problem in lambda calculus with non-idempotent intersection types}}
\author{Ulysse Gérard}
\date\today

\begin{document}

\maketitle

\section{Introduction}
The operation of typing is well known in \lcalc. And numerous typing systems have been imagined to give a signification to \lterms, attributing types to them. But the opposite problem exists, when we want to decide, given a type $A$ and a typing environment $\Gamma$, whether or not there exists a term $t$ such that $\Gamma \vdash t : A$ is provable. This problem is called the \emph{inhabitation problem}.
The inhabitation problem has been studied for numerous type systems and proved to be undecidable for some of them. For example, the system $\mathcal{D}\Omega$ \cite{Kriv} which characterizes solvable terms shown in \ref{SDO} %Th3.10 ?
has been proved to be undecidable by Urzyczyn \cite{Urz}. But others, simplified, type systems such as the  $\mathcal{S}$ and $\mathcal{S_r}$ systems of Van Bakel \cite{VBak} also turned out to be undecidable. All of these system are based on idempotent intersection types ($A\wedge A = A$).

\begin{figure}
    \begin{mdframed}
        \begin{gather*}
        \inference[]{x : A \in \Gamma}{\Gamma \vdash x : A}
        \qquad
        \inference[]{\Gamma, x : A \vdash t : B}{\Gamma \vdash \lambda x.t : A \rightarrow B}
        \qquad
        \inference[]{\Gamma \vdash t : A \rightarrow B \quad \Gamma \vdash u : A}{\Gamma \vdash tu : B}
        \\
        \inference[]{\Gamma t : A \wedge B}{\Gamma \vdash t : A \quad \Gamma t : B}
        \qquad
        \inference[]{\Gamma \vdash t : A \quad \Gamma \vdash t : B}{\Gamma \vdash t : A \wedge B}
        \qquad
        \inference[]{}{\Gamma \vdash t : \Omega}
        \\ \text{For types } A ::= \Omega \ | \ \alpha \ | \ A \rightarrow A \ | \ A \wedge A
        \end{gather*}
    \end{mdframed}
    \caption{System $\mathcal{D}\Omega$}
    \label{SDO}
\end{figure}

Recently, researchers from the PPS laboratory in France and the University of Turin in Italy shoved that the inhabitation problem was decidable for non-idempotent intersection types ($A\wedge A\neq A$) by exhibiting a sound and complete algorithm (Fig. \ref{INH}, p\pageref{INH}) \cite{Kes}. They used a typing system named system $\mathcal{M}$, which is the same as system $\mathcal{R}$ of R. De Carvalho \cite{DeCarvalho}.

The purpose of this work was to implement this algorithm in OCaml.

\begin{figure}[h]
    \begin{mdframed}
        \begin{gather*}
        \inference[(Var)]{}{x : [\rho] \vdash x : \rho}
        \qquad
        \inference[($\rightarrow$ I)]{\Gamma, x : A \vdash t : \rho}{\Gamma \backslash x \vdash \lambda x.t : A \rightarrow \rho}
        \\ \\
        \inference[($\rightarrow$ E)]{\Gamma_0 \vdash t : [\sigma_1, \ldots \sigma_i] \rightarrow \rho 
            & \Gamma_1 \vdash u : \sigma_1, \ldots \Gamma_i \vdash u : \sigma_i}
        {\Gamma_0 + \Gamma_1 + \cdots + \Gamma_i \vdash tu : \rho}
        \\ \\
        \text{For types } \sigma, \tau, \rho ::= \alpha \ | \ A \rightarrow \tau \qquad A ::= [\sigma_i]_{i \in I}
        \\ \text{And where $+$ refers to the \emph{multiset union}, usually denoted $\uplus$.}
        \end{gather*}
    \end{mdframed}
    \caption{The  $\mathcal{M}$-system ($\mathcal{R}$-system of D. De Carvalho \cite{DeCarvalho})}
\end{figure}

%\begin{Def}
%    A term t is \emph{solvable} iff it exists vars $x_1,\ldots,x_k$ and terms $u_1,\ldots,u_k,v_1,\ldots,v_l$ (with $k,l\geq 0$) such that $(t[\sfrac{u_1}{x_1},\ldots,\sfrac{u_k}{x_k}])v_1\ldots v_l \simeq_\beta I$
%\end{Def}

\section {Definitions}
The \lci\ and the non-idempotent intersection types used are:
\begin{align*}
	t, u, v &::= x\ |\ \lambda x.t\ |\ tu 
	& \sigma, \tau, \rho &::= \alpha \ | A \rightarrow \tau \\
	&& A &::= [\sigma_i]_{i \in I}
\end{align*}
where $\alpha$ belongs to a countable set of base types and $I$ is a finite set of indices. Non-idempotent intersection types are therefore represented by multisets.

In the type \sysm, represented in Fig. \ref{SDO}, rules (Var) and ($\rightarrow$ I) are the same than in simply typed $\lambda$-calculus. The third rule is the interesting one, it implements the use of non-idempotent intersection types and allows the system to type even non-strongly normalizing terms when $I = \emptyset$, for example:
\begin{displaymath}
    \inference[($\rightarrow$ E)]{[[] \rightarrow \alpha] \vdash x : [] \rightarrow \rho}{x : [[] \rightarrow \alpha] \vdash x((\lambda y.yy)(\lambda y.yy)) : \alpha}
\end{displaymath}
The term is typed, even if some of its subterms remain untyped. Thus a notation for these untyped subterms is needed in order to reconstruct the subject of a derivation, it is the standard notion of \emph{approximate normal form} (anf):
\begin{align*}
    a,b,c ::= \Omega\ |\ \mathcal{N} 
    && \mathcal{N} ::= \lambda x.\mathcal{N}\ |\ \mathcal{L}
    && \mathcal{L} ::= x\ |\ \mathcal{L}a
\end{align*}
And we write $\leq$ the smallest contextual order such that for all anf a, $\Omega \leq a$. Therefore, $a \leq t$ indicates the fact that t is obtained by replacing all the occurrences of $\Omega$ in a by terms.



\section{The algorithm}
\begin{figure}[h]
    \begin{mdframed}
        \begin{gather*}
            \inference[(ABS)]{a \in \iT(\Gamma + (x : A), \tau) & x \notin \iDom(\Gamma)}{\iLambda x.a \in \iT(\Gamma, A \rightarrow \tau)}
            \\ \\
            \inference[(HEAD)]{a \in \iT^{x:[\sigma]}_x(\Gamma, \sigma) \triangleright \tau}{a \in\iT(\Gamma + (x:[\sigma]), \tau) }
            \\ \\
            \inference[(PREFIX)]{\Gamma = \Gamma_0 + \Gamma_1 & b \in \iTI(\Gamma_0, A) & a \in \iH^{\Delta + \Gamma_0}_{cb}(\Gamma_1, \sigma) \triangleright \tau}{a \in \iH^{\Delta}_c(\Gamma, A \rightarrow \sigma) \triangleright \tau}
            \\ \\
            \inference[(FINAL)]{\sigma = \tau}{a \in \iH^{\Delta}_a(\emptyset, \sigma) \triangleright \tau}
            \qquad
            \inference[(UNION)]{(a_i \in \iT(\Gamma_i, \sigma_i))_{i \in I} & \uparrow_{i \in I}a_i}{\displaystyle \bigvee_{i \in I} a_i \in \iTI(+_{i \in I}\Gamma_i, [\sigma_i]_{i \in I}) }
        \end{gather*}
    \end{mdframed}
  \caption{The inhabitation algorithm}
  \label{INH}
\end{figure}
The algorithm (Fig. \ref{INH}) is separated in three functions, $\iT$, $\iH$ and $\iTI$:
\begin{itemize}
\item $\iT(\Gamma, \sigma)$ contains all the anf a such that there exists a derivation $\Pi \triangleright \Gamma \vdash a : \sigma$ with $a = \mathcal{A}(\Pi)$. To build this set and then stop is the purpose of the inhabitation algorithm.
\item $\iH^\Delta_a(\Gamma, \sigma) \triangleright \tau$ contains all the anf b such that a us a head subterm of b, and such that $a \in \iT(\Delta, \sigma) \Rightarrow b \in \iT(\Gamma + \Delta, \tau)$.
\item $\iTI(\Gamma, [\sigma_i]_{i \in I})$ contains all the anf $a = \bigvee_{i\in I}a_i$ such that $ \Gamma = +_{i \in I} \Gamma_i $, $ a_i \in \iT(\Gamma_i, \sigma_i)\text{ for all }i \in I $ and $\uparrow_{i \in I}a_i$
\end{itemize}

\section{Implementation}
This implementation of the inhabitation algorithm only uses basic OCaml structures. It literally follows the algorithm and builds a list of terms being given an environment and a type. The main algorithm involves a lot of undeterminism and it's implementation is not tail-recursive which makes it very memory greedy.

\subsection{Types}
The implementation of the inhabitation algorithm uses basic OCaml sum types with polymorphism and recursion. 

\
\begin{lstlisting}[caption={Multisets}]
type 'a multiset =
  Empty
| Cons of 'a * 'a multiset
\end{lstlisting}

\begin{lstlisting}[caption={Intersection types and environments}]
type multisetType = 
  sType multiset
and sType = 
  Var of char
| Fleche of multisetType * sType

type environment = (char * multisetType) list
\end{lstlisting}

\begin{lstlisting}[caption={Approximate normal forms}]
type anf = 
  Omega
| N of n
and n = 
  Lambda of char * n
| L of l
  and l =
    Var of char
  | App of l * anf
\end{lstlisting}


\subsection{Tools (Partitions)}
The algorithm uses lot of little helper functions to manipulate the data structures.

In particular, the $\ienvSPlit$ function is able to list all the n-partitions of a given environment using another function, $\isplit$ which returns all the n-partitions of a multiset. It does a massive usage of List.map to iterate the split for each cases. Its complexity lie in the structure of environments which are lists of multisets which all needs to be divided.

%\begin{minipage}{\textwidth}
\begin{lstlisting}[caption={Partitions of an environment}]
let rec envSplitsN n = function
  [] -> [initList n]
  | (x, tys)::tl -> let decoupes = splitsN tys n 
                       and reste = envSplitsN n tl in
    let decoupesX = 
      (* Re-establishing 
          the (x : multiset) list structure *)
      List.map (fun paquets -> 
        List.map (fun paquet -> 
            (x, paquet)
        ) paquets
      ) decoupes in
            
    List.concat (
      (* Adding to the rest all the different
          choices for each var *)
      List.map (fun envs -> 
        (* For each choice of x 
            we divide again the environments *)
        List.map (fun decoupeX -> 
          List.map2 (fun x env -> 
            x::env
          ) decoupeX envs
        ) decoupesX
      ) reste
    )
\end{lstlisting}
%\end{minipage}

\subsection{Inhabitation}
This function is made of numerous sub-routines which reproduce the rules of the algorithm. The undeterminism is materialised by the concatenation of the different branches of the computation.

\section{Termination and soundness}

\section{Conclusion}
    Opening toward pattern matching etc.
\newpage
\bibliographystyle{plain}
\bibliography{main.bib}

\end{document}